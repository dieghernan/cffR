% Options for packages loaded elsewhere
\PassOptionsToPackage{unicode}{hyperref}
\PassOptionsToPackage{hyphens}{url}
\PassOptionsToPackage{dvipsnames,svgnames,x11names}{xcolor}
%
\documentclass[
]{article}
\title{\BibTeX~and CFF}
\usepackage{etoolbox}
\makeatletter
\providecommand{\subtitle}[1]{% add subtitle to \maketitle
  \apptocmd{\@title}{\par {\large #1 \par}}{}{}
}
\makeatother
\subtitle{A potential crosswalk}
\author{Diego Hernangómez}
\date{}

\usepackage{amsmath,amssymb}
\usepackage{lmodern}
\usepackage{iftex}
\ifPDFTeX
  \usepackage[T1]{fontenc}
  \usepackage[utf8]{inputenc}
  \usepackage{textcomp} % provide euro and other symbols
\else % if luatex or xetex
  \usepackage{unicode-math}
  \defaultfontfeatures{Scale=MatchLowercase}
  \defaultfontfeatures[\rmfamily]{Ligatures=TeX,Scale=1}
\fi
% Use upquote if available, for straight quotes in verbatim environments
\IfFileExists{upquote.sty}{\usepackage{upquote}}{}
\IfFileExists{microtype.sty}{% use microtype if available
  \usepackage[]{microtype}
  \UseMicrotypeSet[protrusion]{basicmath} % disable protrusion for tt fonts
}{}
\makeatletter
\@ifundefined{KOMAClassName}{% if non-KOMA class
  \IfFileExists{parskip.sty}{%
    \usepackage{parskip}
  }{% else
    \setlength{\parindent}{0pt}
    \setlength{\parskip}{6pt plus 2pt minus 1pt}}
}{% if KOMA class
  \KOMAoptions{parskip=half}}
\makeatother
\usepackage{xcolor}
\IfFileExists{xurl.sty}{\usepackage{xurl}}{} % add URL line breaks if available
\IfFileExists{bookmark.sty}{\usepackage{bookmark}}{\usepackage{hyperref}}
\hypersetup{
  pdftitle={~and CFF},
  pdfauthor={Diego Hernangómez},
  colorlinks=true,
  linkcolor={brown},
  filecolor={Maroon},
  citecolor={Blue},
  urlcolor={blue},
  pdfcreator={LaTeX via pandoc}}
\urlstyle{same} % disable monospaced font for URLs
\usepackage{color}
\usepackage{fancyvrb}
\newcommand{\VerbBar}{|}
\newcommand{\VERB}{\Verb[commandchars=\\\{\}]}
\DefineVerbatimEnvironment{Highlighting}{Verbatim}{commandchars=\\\{\}}
% Add ',fontsize=\small' for more characters per line
\usepackage{framed}
\definecolor{shadecolor}{RGB}{248,248,248}
\newenvironment{Shaded}{\begin{snugshade}}{\end{snugshade}}
\newcommand{\AlertTok}[1]{\textcolor[rgb]{0.94,0.16,0.16}{#1}}
\newcommand{\AnnotationTok}[1]{\textcolor[rgb]{0.56,0.35,0.01}{\textbf{\textit{#1}}}}
\newcommand{\AttributeTok}[1]{\textcolor[rgb]{0.77,0.63,0.00}{#1}}
\newcommand{\BaseNTok}[1]{\textcolor[rgb]{0.00,0.00,0.81}{#1}}
\newcommand{\BuiltInTok}[1]{#1}
\newcommand{\CharTok}[1]{\textcolor[rgb]{0.31,0.60,0.02}{#1}}
\newcommand{\CommentTok}[1]{\textcolor[rgb]{0.56,0.35,0.01}{\textit{#1}}}
\newcommand{\CommentVarTok}[1]{\textcolor[rgb]{0.56,0.35,0.01}{\textbf{\textit{#1}}}}
\newcommand{\ConstantTok}[1]{\textcolor[rgb]{0.00,0.00,0.00}{#1}}
\newcommand{\ControlFlowTok}[1]{\textcolor[rgb]{0.13,0.29,0.53}{\textbf{#1}}}
\newcommand{\DataTypeTok}[1]{\textcolor[rgb]{0.13,0.29,0.53}{#1}}
\newcommand{\DecValTok}[1]{\textcolor[rgb]{0.00,0.00,0.81}{#1}}
\newcommand{\DocumentationTok}[1]{\textcolor[rgb]{0.56,0.35,0.01}{\textbf{\textit{#1}}}}
\newcommand{\ErrorTok}[1]{\textcolor[rgb]{0.64,0.00,0.00}{\textbf{#1}}}
\newcommand{\ExtensionTok}[1]{#1}
\newcommand{\FloatTok}[1]{\textcolor[rgb]{0.00,0.00,0.81}{#1}}
\newcommand{\FunctionTok}[1]{\textcolor[rgb]{0.00,0.00,0.00}{#1}}
\newcommand{\ImportTok}[1]{#1}
\newcommand{\InformationTok}[1]{\textcolor[rgb]{0.56,0.35,0.01}{\textbf{\textit{#1}}}}
\newcommand{\KeywordTok}[1]{\textcolor[rgb]{0.13,0.29,0.53}{\textbf{#1}}}
\newcommand{\NormalTok}[1]{#1}
\newcommand{\OperatorTok}[1]{\textcolor[rgb]{0.81,0.36,0.00}{\textbf{#1}}}
\newcommand{\OtherTok}[1]{\textcolor[rgb]{0.56,0.35,0.01}{#1}}
\newcommand{\PreprocessorTok}[1]{\textcolor[rgb]{0.56,0.35,0.01}{\textit{#1}}}
\newcommand{\RegionMarkerTok}[1]{#1}
\newcommand{\SpecialCharTok}[1]{\textcolor[rgb]{0.00,0.00,0.00}{#1}}
\newcommand{\SpecialStringTok}[1]{\textcolor[rgb]{0.31,0.60,0.02}{#1}}
\newcommand{\StringTok}[1]{\textcolor[rgb]{0.31,0.60,0.02}{#1}}
\newcommand{\VariableTok}[1]{\textcolor[rgb]{0.00,0.00,0.00}{#1}}
\newcommand{\VerbatimStringTok}[1]{\textcolor[rgb]{0.31,0.60,0.02}{#1}}
\newcommand{\WarningTok}[1]{\textcolor[rgb]{0.56,0.35,0.01}{\textbf{\textit{#1}}}}
\usepackage{longtable,booktabs,array}
\usepackage{calc} % for calculating minipage widths
% Correct order of tables after \paragraph or \subparagraph
\usepackage{etoolbox}
\makeatletter
\patchcmd\longtable{\par}{\if@noskipsec\mbox{}\fi\par}{}{}
\makeatother
% Allow footnotes in longtable head/foot
\IfFileExists{footnotehyper.sty}{\usepackage{footnotehyper}}{\usepackage{footnote}}
\makesavenoteenv{longtable}
\usepackage{graphicx}
\makeatletter
\def\maxwidth{\ifdim\Gin@nat@width>\linewidth\linewidth\else\Gin@nat@width\fi}
\def\maxheight{\ifdim\Gin@nat@height>\textheight\textheight\else\Gin@nat@height\fi}
\makeatother
% Scale images if necessary, so that they will not overflow the page
% margins by default, and it is still possible to overwrite the defaults
% using explicit options in \includegraphics[width, height, ...]{}
\setkeys{Gin}{width=\maxwidth,height=\maxheight,keepaspectratio}
% Set default figure placement to htbp
\makeatletter
\def\fps@figure{htbp}
\makeatother
\setlength{\emergencystretch}{3em} % prevent overfull lines
\providecommand{\tightlist}{%
  \setlength{\itemsep}{0pt}\setlength{\parskip}{0pt}}
\setcounter{secnumdepth}{-\maxdimen} % remove section numbering
\newlength{\cslhangindent}
\setlength{\cslhangindent}{1.5em}
\newlength{\csllabelwidth}
\setlength{\csllabelwidth}{3em}
\newlength{\cslentryspacingunit} % times entry-spacing
\setlength{\cslentryspacingunit}{\parskip}
\newenvironment{CSLReferences}[2] % #1 hanging-ident, #2 entry spacing
 {% don't indent paragraphs
  \setlength{\parindent}{0pt}
  % turn on hanging indent if param 1 is 1
  \ifodd #1
  \let\oldpar\par
  \def\par{\hangindent=\cslhangindent\oldpar}
  \fi
  % set entry spacing
  \setlength{\parskip}{#2\cslentryspacingunit}
 }%
 {}
\usepackage{calc}
\newcommand{\CSLBlock}[1]{#1\hfill\break}
\newcommand{\CSLLeftMargin}[1]{\parbox[t]{\csllabelwidth}{#1}}
\newcommand{\CSLRightInline}[1]{\parbox[t]{\linewidth - \csllabelwidth}{#1}\break}
\newcommand{\CSLIndent}[1]{\hspace{\cslhangindent}#1}

\def\BibTeX{{\rm B\kern-.05em{\sc i\kern-.025em b}\kern-.08em
    T\kern-.1667em\lower.7ex\hbox{E}\kern-.125emX}}
\usepackage{fvextra} \DefineVerbatimEnvironment{Highlighting}{Verbatim}{breaklines,commandchars=\\\{\}}
\ifLuaTeX
  \usepackage{selnolig}  % disable illegal ligatures
\fi

\begin{document}
\maketitle
\begin{abstract}
This article presents a crosswalk between \BibTeX~and Citation File
Format (\protect\hyperlink{ref-druskat_citation_2021}{Druskat et al.
2021}), as it is performed by the cffr package
(\protect\hyperlink{ref-hernangomez2021}{Hernangómez 2021}).
\end{abstract}

\hypertarget{citation}{%
\subsection{Citation}\label{citation}}

Please cite this article as:

Hernangómez D (2022). ``BibTeX and CFF, a potential crosswalk.''
\emph{The cffr package}, \emph{Vignettes}.

A \BibTeX~entry for \LaTeX~users:

\begin{Shaded}
\begin{Highlighting}[]
\VariableTok{@article}\NormalTok{\{}\OtherTok{hernangomez2022}\NormalTok{,}
    \DataTypeTok{title}\NormalTok{        = \{\{BibTeX\} and \{CFF\}, a potential crosswalk\},}
    \DataTypeTok{author}\NormalTok{       = \{Diego Hernangómez\},}
    \DataTypeTok{year}\NormalTok{         = 2022,}
    \DataTypeTok{journal}\NormalTok{      = \{The \{cffr\} package\},}
    \DataTypeTok{volume}\NormalTok{       = \{Vignettes\}}
\NormalTok{\}}
\end{Highlighting}
\end{Shaded}

\hypertarget{and-r}{%
\subsection{\texorpdfstring{\BibTeX~and R}{~and R}}\label{and-r}}

\href{https://en.wikipedia.org/wiki/BibTeX}{\BibTeX} is a well-known
format for storing references that may be reused by another software,
like {[}\LaTeX{]}(\url{https://en.wikipedia.org/wiki/LaTeX} ``LaTeX'').
\BibTeX~was created by
\href{https://en.wikipedia.org/wiki/Oren_Patashnik}{Oren Patashnik} and
\href{https://en.wikipedia.org/wiki/Leslie_Lamport}{Leslie Lamport} back
in 1985. An example structure of a \BibTeX~entry would be:

\begin{Shaded}
\begin{Highlighting}[]
\VariableTok{@book}\NormalTok{\{}\OtherTok{einstein1921}\NormalTok{,}
    \DataTypeTok{title}\NormalTok{        = \{Relativity: The Special and the General Theory\},}
    \DataTypeTok{author}\NormalTok{       = \{Einstein, A.\},}
    \DataTypeTok{year}\NormalTok{         = 1920,}
    \DataTypeTok{publisher}\NormalTok{    = \{Henry Holt and Company\},}
    \DataTypeTok{address}\NormalTok{      = \{London, United Kingdom\},}
    \DataTypeTok{isbn}\NormalTok{         = 9781587340925}
\NormalTok{\}}
\end{Highlighting}
\end{Shaded}

On this case, the entry (identified as \texttt{einstein1921}) would
refer to a book. This entry then can be used on a document and include
references to it.

On \textbf{R} (\protect\hyperlink{ref-R_2021}{R Core Team 2021}), we can
replicate this structure using the \texttt{bibentry()} and
\texttt{toBibtex()} functions:

\begin{Shaded}
\begin{Highlighting}[]

\NormalTok{entry }\OtherTok{\textless{}{-}} \FunctionTok{bibentry}\NormalTok{(}\StringTok{"Book"}\NormalTok{,}
  \AttributeTok{key =} \StringTok{"einstein1921"}\NormalTok{,}
  \AttributeTok{title =} \StringTok{"Relativity: The Special and the General Theory"}\NormalTok{,}
  \AttributeTok{author =} \FunctionTok{person}\NormalTok{(}\StringTok{"A."}\NormalTok{, }\StringTok{"Einstein"}\NormalTok{),}
  \AttributeTok{year =} \DecValTok{1920}\NormalTok{,}
  \AttributeTok{publisher =} \StringTok{"Henry Holt and Company"}\NormalTok{,}
  \AttributeTok{address =} \StringTok{"London, United Kingdom"}\NormalTok{,}
  \AttributeTok{isbn =} \DecValTok{9781587340925}\NormalTok{,}
\NormalTok{)}

\FunctionTok{toBibtex}\NormalTok{(entry)}
\SpecialCharTok{@}\NormalTok{Book\{einstein1921,}
\NormalTok{  title }\OtherTok{=}\NormalTok{ \{Relativity}\SpecialCharTok{:}\NormalTok{ The Special and the General Theory\},}
\NormalTok{  author }\OtherTok{=}\NormalTok{ \{A. Einstein\},}
\NormalTok{  year }\OtherTok{=}\NormalTok{ \{}\DecValTok{1920}\NormalTok{\},}
\NormalTok{  publisher }\OtherTok{=}\NormalTok{ \{Henry Holt and Company\},}
\NormalTok{  address }\OtherTok{=}\NormalTok{ \{London, United Kingdom\},}
\NormalTok{  isbn }\OtherTok{=}\NormalTok{ \{}\DecValTok{9781587340925}\NormalTok{\},}
\NormalTok{\}}
\end{Highlighting}
\end{Shaded}

The final results of the entry as a text string would be parsed
as\footnote{By default R Pandoc would generate the cite on the Chicago
  author-date format (\protect\hyperlink{ref-rmarkdowncookbook2020}{Xie,
  Dervieux, and Riederer 2020})}:

Einstein A (1920). \emph{Relativity: The Special and the General
Theory}. Henry Holt and Company, London, United Kingdom. ISBN
9781587340925.

\hypertarget{definitions}{%
\subsection{\texorpdfstring{\BibTeX~definitions}{~definitions}}\label{definitions}}

\protect\hyperlink{ref-patashnik1988}{Patashnik}
(\protect\hyperlink{ref-patashnik1988}{1988}) provides a comprehensive
explanation of the \BibTeX~formats. We can distinguish between
\textbf{Entries} and \textbf{Fields}.

\hypertarget{entries}{%
\subsubsection{Entries}\label{entries}}

Each entry type defines a different type of work. The 14 entry types
defined on \BibTeX~\footnote{Other implementations similar to \BibTeX,
  as \href{https://www.ctan.org/pkg/biblatex}{BibLaTeX}, expand the
  definitions of entries including other types as \textbf{online},
  \textbf{software} or \textbf{dataset}. On \BibTeX~these entries should
  be reclassified to \textbf{misc}.} are:

\begin{itemize}
\tightlist
\item
  \textbf{@article}: An article from a journal or magazine.
\item
  \textbf{@book}: A book with an explicit publisher.
\item
  \textbf{@booklet}: A work that is printed and bound, but without a
  named publisher or sponsoring institution.
\item
  \textbf{@conference}: The same as \textbf{@inproceedings}, included
  for Scribe compatibility.
\item
  \textbf{@inbook}: A part of a book, which may be a chapter (or section
  or whatever) and/or a range of pages.
\item
  \textbf{@incollection}: A part of a book having its own title.
\item
  \textbf{@inproceedings}: An article in a conference proceedings.
\item
  \textbf{@manual}: Technical documentation.
\item
  \textbf{@mastersthesis}: A Master's thesis.
\item
  \textbf{@misc}: Use this type when nothing else fits.
\item
  \textbf{@phdthesis}: A PhD thesis.
\item
  \textbf{@proceedings}: The proceedings of a conference.
\item
  \textbf{@techreport}: A report published by a school or other
  institution, usually numbered within a series.
\item
  \textbf{@unpublished}: A document having an author and title, but not
  formally published.
\end{itemize}

Regarding the entries, \texttt{bibentry()} \textbf{R} function does not
implement \textbf{@conference} . However, we can replace that key by
\textbf{@inproceedings} given that the definition is identical.

\hypertarget{fields}{%
\subsubsection{Fields}\label{fields}}

As in the case of Entries,
\protect\hyperlink{ref-patashnik1988}{Patashnik}
(\protect\hyperlink{ref-patashnik1988}{1988}) provides also a definition
for each of the possible standard \BibTeX~fields\footnote{As in the case
  of the entries, other implementations based on BibTeX may recognize
  additional fields.}. An entry can include other fields that would be
ignored on the raw implementation of \BibTeX:

\begin{itemize}
\tightlist
\item
  \textbf{address}: Usually the address of the \textbf{publisher} or
  other of \textbf{institution}.
\item
  \textbf{annote}: An annotation. It is not used by the standard
  bibliography styles, but may be used by others that produce an
  annotated bibliography.
\item
  \textbf{author}: The name(s) of the author(s), in the format described
  n the \LaTeX book (\protect\hyperlink{ref-lamport86latex}{Lamport
  1986}).
\item
  \textbf{booktitle}: Title of a book, part of which is being cited. For
  \textbf{@book} entries, use the \textbf{title} field instead.
\item
  \textbf{chapter}: A chapter (or section or whatever) number.
\item
  \textbf{crossref}: The database key of the entry being cross
  referenced.
\item
  \textbf{edition}: The edition of a \textbf{@book} - for example,
  ``Second.'' This should be an ordinal, and should have the first
  letter capitalized, the standard styles convert to lower case when
  necessary.
\item
  \textbf{editor}: Name(s) of editor(s), typed as indicated in the LaTeX
  book (\protect\hyperlink{ref-lamport86latex}{Lamport 1986}). If there
  is also an \textbf{author} field, then the editor field gives the
  editor of the book or collection in which the reference appears.
\item
  \textbf{howpublished}: How something strange has been published. The
  first word should be capitalized.
\item
  \textbf{institution}: The sponsoring institution of a technical
  report.
\item
  \textbf{journal}: A journal name.
\item
  \textbf{key}: Used for alphabetizing, cross referencing, and creating
  a label when the \textbf{author} information is missing.
\item
  \textbf{month}: The month in which the work was published or, for an
  unpublished work, in which it was written. You should use the standard
  three-letter abbreviation, as described in Appendix B.1.3 of the LaTeX
  book (\protect\hyperlink{ref-lamport86latex}{Lamport 1986})
  (i.e.~\texttt{jan,\ feb,\ mar}).
\item
  \textbf{note}: Any additional information that can help the reader.
  The first word should be capitalized.
\item
  \textbf{number}: The number of a journal, magazine, technical report,
  or of a work in a series. An issue of a journal or magazine is usually
  identified by its \textbf{volume}: and number; the organization that
  issues a technical report usually gives it a number; and sometimes
  books are given numbers in a named series.
\item
  \textbf{organization}: The organization that sponsors a
  \textbf{@conference} or that publishes a manual.
\item
  \textbf{pages}: One or more page numbers or range of numbers, such as
  \texttt{42-\/-111} or \texttt{7,41,73-\/-97} or \texttt{43+}.
\item
  \textbf{publisher}: The publisher's name.
\item
  \textbf{school}: The name of the school where a thesis was written.
\item
  \textbf{series}: The name of a series or set of books. When citing an
  entire book, the \textbf{title} field gives its title and an optional
  \textbf{series} field gives the name of a series or multi-volume set
  in which the book is published.
\item
  \textbf{title}: The work's title.
\item
  \textbf{type}: The type of a technical report---for example,
  ``Research Note.''
\item
  \textbf{volume}: The volume of a journal or multivolume book.
\item
  \textbf{year}: The year of publication or, for an unpublished work,
  the year it was written. Generally it should consist of four numerals,
  such as \texttt{1984}.
\end{itemize}

There is a strict relation between Entries and Fields on BibTeX.
Depending on the type of entries, some fields are required while others
are optional or even ignored. On the following table, required field are
flagged as \textbf{*} and optional fields are flagged as \textbf{-}.
Fields on parenthesis \textbf{()} denotes that there are some degree of
optionality on the requirement of the field, see
\protect\hyperlink{ref-patashnik1988}{Patashnik}
(\protect\hyperlink{ref-patashnik1988}{1988}) for more information.

\begin{longtable}[]{@{}lcccccc@{}}
\caption{BibTeX, required fields by entry}\tabularnewline
\toprule
field & article & book & booklet & inbook & incollection &
conference,inproceedings \\
\midrule
\endfirsthead
\toprule
field & article & book & booklet & inbook & incollection &
conference,inproceedings \\
\midrule
\endhead
title & x & x & x & x & x & x \\
year & x & x & o & x & x & x \\
author & x & (x) & o & (x) & x & x \\
note & o & o & o & o & o & o \\
month & o & o & o & o & o & o \\
address & & o & o & o & o & o \\
publisher & & x & & x & x & o \\
editor & & (x) & & (x) & o & o \\
number & o & (o) & & (o) & (o) & (o) \\
volume & o & (o) & & (o) & (o) & (o) \\
pages & o & & & (x) & o & o \\
series & & o & & o & o & o \\
booktitle & & & & & x & x \\
edition & & o & & o & o & \\
type & & & & o & o & \\
chapter & & & & (x) & o & \\
organization & & & & & & o \\
howpublished & & & o & & & \\
institution & & & & & & \\
journal & x & & & & & \\
school & & & & & & \\
annote & & & & & & \\
crossref & & & & & & \\
key & & & & & & \\
\bottomrule
\end{longtable}

\begin{longtable}[]{@{}lcccccc@{}}
\caption{(cont) BibTeX, required fields by entry}\tabularnewline
\toprule
field & manual & mastersthesis,phdthesis & misc & proceedings &
techreport & unpublished \\
\midrule
\endfirsthead
\toprule
field & manual & mastersthesis,phdthesis & misc & proceedings &
techreport & unpublished \\
\midrule
\endhead
title & x & x & o & x & x & x \\
year & o & x & o & x & x & o \\
author & o & x & o & & x & x \\
note & o & o & o & o & o & x \\
month & o & o & o & o & o & o \\
address & o & o & & o & o & \\
publisher & & & & o & & \\
editor & & & & o & & \\
number & & & & (o) & o & \\
volume & & & & (o) & & \\
pages & & & & & & \\
series & & & & o & & \\
booktitle & & & & & & \\
edition & o & & & & & \\
type & & o & & & o & \\
chapter & & & & & & \\
organization & o & & & o & & \\
howpublished & & & o & & & \\
institution & & & & & x & \\
journal & & & & & & \\
school & & x & & & & \\
annote & & & & & & \\
crossref & & & & & & \\
key & & & & & & \\
\bottomrule
\end{longtable}

It can be observed that just a subset of fields is required in any of
the Entries. For example, \textbf{title}, \textbf{year} and
\textbf{author} are either required or optional on almost every entry,
while \textbf{crossref}, \textbf{annote} or \textbf{key} are never
required.

\hypertarget{citation-file-format}{%
\subsection{Citation File Format}\label{citation-file-format}}

\href{https://citation-file-format.github.io/}{Citation File Format
(CFF}) (\protect\hyperlink{ref-druskat_citation_2021}{Druskat et al.
2021}) are plain text files with human- and machine-readable citation
information for software (and datasets). Among the
\href{https://github.com/citation-file-format/citation-file-format/blob/main/schema-guide.md\#valid-keys}{valid
keys of CFF} there are two keys, \texttt{preferred-citation} and
\texttt{references} of special interest for citing and referring to
related works\footnote{See
  \href{https://github.com/citation-file-format/citation-file-format/blob/main/schema-guide.md\#preferred-citation}{Guide
  to Citation File Format schema version 1.2.0}.}:

\begin{itemize}
\item
  \textbf{\texttt{preferred-citation}}: A reference to another work that
  should be cited instead of the software or dataset itself.
\item
  \textbf{\texttt{references}}: Reference(s) to other creative works.
  Similar to a list of references in a paper, references of the software
  or dataset may include other software (dependencies), or other
  research products that the software or dataset builds on, but not work
  describing the software or dataset.
\end{itemize}

These two keys are expected to be
\href{https://github.com/citation-file-format/citation-file-format/blob/main/schema-guide.md\#definitionsreference}{\texttt{definition.reference}}
objects, therefore they may contain the following keys:

\begin{longtable}[]{@{}lllll@{}}
\caption{Valid keys on CFF \texttt{definition-reference}
objects}\tabularnewline
\toprule
\endhead
abbreviation & abstract & authors & collection-doi & collection-title \\
collection-type & commit & conference & contact & copyright \\
data-type & database-provider & database & date-accessed &
date-downloaded \\
date-published & date-released & department & doi & edition \\
editors & editors-series & end & entry & filename \\
format & identifiers & institution & isbn & issn \\
issue & issue-date & issue-title & journal & keywords \\
languages & license & license-url & loc-end & loc-start \\
location & medium & month & nihmsid & notes \\
number & number-volumes & pages & patent-states & pmcid \\
publisher & recipients & repository & repository-artifact &
repository-code \\
scope & section & senders & start & status \\
term & thesis-type & title & translators & type \\
url & version & volume & volume-title & year \\
year-original & & & & \\
\bottomrule
\end{longtable}

These keys are the equivalent to the fields of BibTeX (see
\protect\hyperlink{fields}{Fields}), with the exception of the key
\textbf{type}. On CFF, this key defines the type of work\footnote{See a
  complete list of possible values of CFF type on the
  \href{https://github.com/citation-file-format/citation-file-format/blob/main/schema-guide.md\#definitionsreferencetype}{Guide
  to Citation File Format schema version 1.2.0}.}, therefore this is the
equivalent to the BibteX entries (see
\protect\hyperlink{entries}{Entries}).

\hypertarget{proposed-crosswalk}{%
\subsection{Proposed crosswalk}\label{proposed-crosswalk}}

The \textbf{cffr} package
(\protect\hyperlink{ref-hernangomez2021}{Hernangómez 2021}) provides
utilities from converting BibTeX entries (via the \textbf{R} base
function \texttt{bibentry()}) to CFF files and vice-versa. This section
describes how the conversion between both formats have been implemented.
This crosswalk is based partially on
\protect\hyperlink{ref-Haines_Ruby_CFF_Library_2021}{Haines and The Ruby
Citation File Format Developers}
(\protect\hyperlink{ref-Haines_Ruby_CFF_Library_2021}{2021})\footnote{Note
  that this software performs only the conversion from CFF to BibTeX,
  however \textbf{cffr} can perform the conversion in both directions.}.

On the following two section I present an overview of the proposed
mapping between the Entries and Fields of BibTeX and the CFF keys. After
this initial mapping, I propose further transformations to improve the
compatibility between both systems using different
\protect\hyperlink{entry-models}{Entry Models}.

\hypertarget{entrytype-crosswalk}{%
\subsubsection{Entry/Type crosswalk}\label{entrytype-crosswalk}}

For converting general BibTeX entries to CFF types, the following
crosswalk is proposed:

\begin{longtable}[]{@{}lll@{}}
\caption{Entry/Type crosswalk: From BibteX to CFF}\tabularnewline
\toprule
BibTeX Entry & Value of CFF key: type & Notes \\
\midrule
\endfirsthead
\toprule
BibTeX Entry & Value of CFF key: type & Notes \\
\midrule
\endhead
\textbf{@article} & article & \\
\textbf{@book} & book & \\
\textbf{@booklet} & pamphlet & \\
\textbf{@conference} & conference-paper & \\
\textbf{@inbook} & book & See \protect\hyperlink{entry-models}{Entry
Models} \\
\textbf{@incollection} & generic & See
\protect\hyperlink{entry-models}{Entry Models} \\
\textbf{@inproceedings} & conference-paper & \\
\textbf{@manual} & manual & \\
\textbf{@mastersthesis} & thesis & See
\protect\hyperlink{entry-models}{Entry Models} \\
\textbf{@misc} & generic & \\
\textbf{@phdthesis} & thesis & See
\protect\hyperlink{entry-models}{Entry Models} \\
\textbf{@proceedings} & proceedings & \\
\textbf{@techreport} & report & \\
\textbf{@unpublished} & unpublished & \\
\bottomrule
\end{longtable}

Also, given that CFF provides with a
\href{https://github.com/citation-file-format/citation-file-format/blob/main/schema-guide.md\#definitionsreferencetype}{wide
range of allowed values} on type, the following conversion would be
performed from CFF to BibTeX:

\begin{longtable}[]{@{}
  >{\raggedright\arraybackslash}p{(\columnwidth - 4\tabcolsep) * \real{0.24}}
  >{\raggedright\arraybackslash}p{(\columnwidth - 4\tabcolsep) * \real{0.32}}
  >{\raggedright\arraybackslash}p{(\columnwidth - 4\tabcolsep) * \real{0.44}}@{}}
\caption{Entry/Type crosswalk: From CFF to BibteX}\tabularnewline
\toprule
\begin{minipage}[b]{\linewidth}\raggedright
Value of CFF key: type
\end{minipage} & \begin{minipage}[b]{\linewidth}\raggedright
BibTeX Entry
\end{minipage} & \begin{minipage}[b]{\linewidth}\raggedright
Notes
\end{minipage} \\
\midrule
\endfirsthead
\toprule
\begin{minipage}[b]{\linewidth}\raggedright
Value of CFF key: type
\end{minipage} & \begin{minipage}[b]{\linewidth}\raggedright
BibTeX Entry
\end{minipage} & \begin{minipage}[b]{\linewidth}\raggedright
Notes
\end{minipage} \\
\midrule
\endhead
book & \textbf{@book} \textbf{/ @inbook} & See
\protect\hyperlink{entry-models}{Entry Models} \\
conference & \textbf{@inproceedings} & \\
conference-paper & \textbf{@inproceedings} & \\
magazine-article & \textbf{@article} & \\
manual & \textbf{@manual} & \\
n ewspaper-article & \textbf{@article} & \\
pamphlet & \textbf{@booklet} & \\
proceedings & \textbf{@proceedings} & \\
report & \textbf{@techreport} & \\
thesis & \textbf{@mastersthesis / @phdthesis} & See
\protect\hyperlink{entry-models}{Entry Models} \\
unpublished & \textbf{@unpublished} & \\
generic & \textbf{@misc / @incollection} & Under specific conditions,
see \protect\hyperlink{entry-models}{Entry Models} \\
\textless any other value\textgreater{} & \textbf{@misc} & \\
\bottomrule
\end{longtable}

\hypertarget{fieldskey-crosswalk}{%
\subsubsection{Fields/Key crosswalk}\label{fieldskey-crosswalk}}

There is a large degree of similarity between the definition and names
of some BibTeX fields and CFF keys. On the following cases, the
equivalence is almost straightforward:

\begin{longtable}[]{@{}ll@{}}
\caption{BibTeX - CFF Field/Key crosswalk}\tabularnewline
\toprule
BibTeX Field & CFF key \\
\midrule
\endfirsthead
\toprule
BibTeX Field & CFF key \\
\midrule
\endhead
\textbf{address} & See \protect\hyperlink{entry-models}{Entry Models} \\
\textbf{\emph{annote}} & - \\
\textbf{author} & authors \\
\textbf{booktitle} & collection-title \\
\textbf{chapter} & section \\
\textbf{\emph{crossref}} & - \\
\textbf{edition} & edition \\
\textbf{editor} & editors \\
\textbf{howpublished} & medium \\
\textbf{institution} & See \protect\hyperlink{entry-models}{Entry
Models} \\
\textbf{journal} & journal \\
\textbf{\emph{key}} & - \\
\textbf{month} & month \\
\textbf{note} & notes \\
\textbf{number} & issue \\
\textbf{organization} & See \protect\hyperlink{entry-models}{Entry
Models} \\
\textbf{pages} & start \& end \\
\textbf{publisher} & publisher \\
\textbf{school} & See \protect\hyperlink{entry-models}{Entry Models} \\
\textbf{series} & See \protect\hyperlink{entry-models}{Entry Models} \\
\textbf{title} & title \\
\textbf{\emph{type}} & - \\
\textbf{volume} & volume \\
\textbf{year} & year \\
\bottomrule
\end{longtable}

Additionally, there are other additional CFF keys that have a
correspondence with BibLaTeX fields. We propose also to include these
fields on the crosswalk\footnote{See
  \protect\hyperlink{ref-biblatexcheatsheet}{Rees}
  (\protect\hyperlink{ref-biblatexcheatsheet}{2017}) for a preview of
  the accepted BibLaTeX fields.}, although they are not part of the core
BibTeX fields definition.

\begin{longtable}[]{@{}ll@{}}
\caption{BibLaTeX - CFF Field/Key crosswalk}\tabularnewline
\toprule
BibLaTeX Field & CFF key \\
\midrule
\endfirsthead
\toprule
BibLaTeX Field & CFF key \\
\midrule
\endhead
\textbf{abstract} & abstract \\
\textbf{date} & date-published \\
\textbf{doi} & doi \\
\textbf{file} & filename \\
\textbf{isbn} & isbn \\
\textbf{issn} & issn \\
\textbf{issuetitle} & issue-title \\
\textbf{pagetotal} & pages \\
\textbf{translator} & translators \\
\textbf{url} & url \\
\textbf{urldate} & date-accessed \\
\textbf{version} & version \\
\bottomrule
\end{longtable}

\hypertarget{entry-models}{%
\subsection{Entry Models}\label{entry-models}}

This section presents the specific mapping proposed for each of the
BibTeX entries, providing further information on how each field is
treated.

\hypertarget{article}{%
\subsubsection{Article}\label{article}}

The crosswalk of @Article does not require any special treatment.

\begin{longtable}[]{@{}
  >{\raggedright\arraybackslash}p{(\columnwidth - 4\tabcolsep) * \real{0.23}}
  >{\raggedright\arraybackslash}p{(\columnwidth - 4\tabcolsep) * \real{0.23}}
  >{\raggedright\arraybackslash}p{(\columnwidth - 4\tabcolsep) * \real{0.54}}@{}}
\caption{Article model}\tabularnewline
\toprule
\begin{minipage}[b]{\linewidth}\raggedright
BibTeX
\end{minipage} & \begin{minipage}[b]{\linewidth}\raggedright
CFF
\end{minipage} & \begin{minipage}[b]{\linewidth}\raggedright
Note
\end{minipage} \\
\midrule
\endfirsthead
\toprule
\begin{minipage}[b]{\linewidth}\raggedright
BibTeX
\end{minipage} & \begin{minipage}[b]{\linewidth}\raggedright
CFF
\end{minipage} & \begin{minipage}[b]{\linewidth}\raggedright
Note
\end{minipage} \\
\midrule
\endhead
\textbf{@Article} & article & When converting CFF to BibTeX, CFF keys
\emph{magazine-article} and \emph{newspaper-article} are converted to
\textbf{@Article}. \\
\textbf{author*} & authors & \\
\textbf{title*} & title & \\
\textbf{journal*} & journal & \\
\textbf{year*} & year & \\
\textbf{volume} & volume & \\
\textbf{number} & issue & \\
\textbf{pages} & start and end & Separated by \texttt{-\/-,} i.e,

pages = \texttt{3-\/-5} (BibTeX)

would be parsed as

\texttt{start:\ 3}

\texttt{end:\ 5}

on CFF \\
\textbf{month} & month & As a fallback, \textbf{month} could be
extracted also from \textbf{date} (BibLaTeX field)/ date-published
(CFF) \\
\textbf{note} & notes & \\
\bottomrule
\end{longtable}

\textbf{Examples}

\underline{\emph{BibTeX entry}}

\begin{Shaded}
\begin{Highlighting}[]
\VariableTok{@Article}\NormalTok{\{}\OtherTok{1981:surveys:bernstein}\NormalTok{,}
  \DataTypeTok{author}\NormalTok{ =       "}\StringTok{P. A. Bernstein and N. Goldman}\NormalTok{",}
  \DataTypeTok{title}\NormalTok{ =        "}\StringTok{Concurrency Control in Distributed Database Systems}\NormalTok{",}
  \DataTypeTok{journal}\NormalTok{ =      "}\StringTok{ACM Computing Surveys}\NormalTok{",}
  \DataTypeTok{publisher}\NormalTok{ =    "}\StringTok{ACM Press}\NormalTok{",}
  \DataTypeTok{address}\NormalTok{ =      "}\StringTok{New York, NY, USA}\NormalTok{",}
  \DataTypeTok{year}\NormalTok{ =         "}\StringTok{1981}\NormalTok{",}
  \DataTypeTok{volume}\NormalTok{ =       "}\StringTok{13}\NormalTok{",}
  \DataTypeTok{month}\NormalTok{ =        "}\StringTok{jun}\NormalTok{",}
  \DataTypeTok{number}\NormalTok{ =       "}\StringTok{2}\NormalTok{",}
  \DataTypeTok{pages}\NormalTok{ =        "}\StringTok{185{-}{-}221}\NormalTok{",}
  \DataTypeTok{keywords}\NormalTok{ =     "}\StringTok{Concurrency control; deadlock; distributed database}
\StringTok{                 management systems; locking; serializability;}
\StringTok{                 synchronization; timestamp ordering; timestamps;}
\StringTok{                 two{-}phase commit; two{-}phase locking}\NormalTok{",}
  \DataTypeTok{annote}\NormalTok{ =       "}\StringTok{unsupported}\NormalTok{",}
\NormalTok{\}}
\end{Highlighting}
\end{Shaded}

\underline{\emph{CFF entry}}

\begin{verbatim}
type: article
title: Concurrency Control in Distributed Database Systems
authors:
- family-names: Bernstein
  given-names: P. A.
- family-names: Goldman
  given-names: N.
journal: ACM Computing Surveys
publisher:
  name: ACM Press
  address: New York, NY, USA
year: '1981'
volume: '13'
month: '6'
issue: '2'
keywords:
- Concurrency control
- deadlock
- distributed database management systems
- locking
- serializability
- synchronization
- timestamp ordering
- timestamps
- two-phase commit
- two-phase locking
start: '185'
end: '221'
\end{verbatim}

\underline{\emph{From CFF to BibTeX}}

\begin{verbatim}
@Article{bernsteingoldman:1981,
  title = {Concurrency Control in Distributed Database Systems},
  author = {P. A. Bernstein and N. Goldman},
  year = {1981},
  month = {jun},
  journal = {ACM Computing Surveys},
  publisher = {ACM Press},
  address = {New York, NY, USA},
  volume = {13},
  number = {2},
  pages = {185--221},
  keywords = {Concurrency control, deadlock, distributed database management systems, locking, serializability, synchronization, timestamp ordering, timestamps, two-phase commit, two-phase locking},
}
\end{verbatim}

\hypertarget{bookinbook}{%
\subsubsection{Book/InBook}\label{bookinbook}}

In terms of field required on BibTeX, the only difference between
\textbf{@Book} and \textbf{@InBook} is that the latter requires also a
\textbf{chapter} or \textbf{pages}, while for @Book these fields are not
even optional. So we propose here to identify an \textbf{@InBook} on CFF
as a book with chapter and start-end fields (CFF).

Another specificities are that \textbf{series} field is mapped to
collection-title and \textbf{address} is mapped as the address of the
publisher (CFF).

\begin{longtable}[]{@{}
  >{\raggedright\arraybackslash}p{(\columnwidth - 4\tabcolsep) * \real{0.27}}
  >{\raggedright\arraybackslash}p{(\columnwidth - 4\tabcolsep) * \real{0.24}}
  >{\raggedright\arraybackslash}p{(\columnwidth - 4\tabcolsep) * \real{0.49}}@{}}
\caption{Book/InBook model}\tabularnewline
\toprule
\begin{minipage}[b]{\linewidth}\raggedright
BibTeX
\end{minipage} & \begin{minipage}[b]{\linewidth}\raggedright
CFF
\end{minipage} & \begin{minipage}[b]{\linewidth}\raggedright
Note
\end{minipage} \\
\midrule
\endfirsthead
\toprule
\begin{minipage}[b]{\linewidth}\raggedright
BibTeX
\end{minipage} & \begin{minipage}[b]{\linewidth}\raggedright
CFF
\end{minipage} & \begin{minipage}[b]{\linewidth}\raggedright
Note
\end{minipage} \\
\midrule
\endhead
\textbf{@Book} & book & \\
\textbf{@InBook} & book & For identifying an @InBook in CFF, assess if
\textbf{chapter} or \textbf{pages}/start-end information is available \\
\textbf{author*} & authors & \\
\textbf{editor*} & editors & \\
\textbf{title*} & title & \\
\textbf{publisher*} & publisher & \\
\textbf{year*} & year & \\
\textbf{chapter*} & section & Only required on \textbf{@InBook} \\
\textbf{pages*} & start and end & Only required on \textbf{@InBook} \\
\textbf{volume} & volume & \\
\textbf{number} & issue & \\
\textbf{series} & colle ction-title & \\
\textbf{address} & address property of publisher & As a fallback, the
field location (CFF) can be used \\
\textbf{edition} & edition & \\
\textbf{month} & month & As a fallback, \textbf{month} could be
extracted also from \textbf{date} (BibLaTeX field)/ date-published
(CFF) \\
\textbf{note} & notes & \\
\bottomrule
\end{longtable}

\textbf{Examples: Book}

\underline{\emph{BibTeX entry}}

\begin{Shaded}
\begin{Highlighting}[]
\VariableTok{@book}\NormalTok{\{}\OtherTok{10}\NormalTok{.}\OtherTok{5555}\NormalTok{/}\OtherTok{218623}\NormalTok{,}
    \DataTypeTok{editor}\NormalTok{       = \{Abramsky, S. and Gabbay, Dov M. and Maibaum, T. S. E.\},}
    \DataTypeTok{title}\NormalTok{        = \{Handbook of Logic in Computer Science: Semantic Modelling\},}
    \DataTypeTok{year}\NormalTok{         = 1995,}
    \DataTypeTok{volume}\NormalTok{       = 4,}
    \DataTypeTok{isbn}\NormalTok{         = \{0198537808\},}
    \DataTypeTok{publisher}\NormalTok{    = \{Oxford University Press, Inc.\},}
    \DataTypeTok{address}\NormalTok{      = \{USA\}}
\NormalTok{\}}
\end{Highlighting}
\end{Shaded}

\underline{\emph{CFF entry}}

\begin{verbatim}
type: book
title: 'Handbook of Logic in Computer Science: Semantic Modelling'
authors:
- name: anonymous
editors:
- family-names: Abramsky
  given-names: S.
- family-names: Gabbay
  given-names: Dov M.
- family-names: Maibaum
  given-names: T. S. E.
volume: '4'
year: '1995'
isbn: 0198537808
publisher:
  name: Oxford University Press, Inc.
  address: USA
\end{verbatim}

\underline{\emph{From CFF to BibTeX}}

\begin{Shaded}
\begin{Highlighting}[]
\FunctionTok{toBibtex}\NormalTok{(}\FunctionTok{cff\_to\_bibtex}\NormalTok{(}\FunctionTok{cff\_parse\_citation}\NormalTok{(bib)))}
\SpecialCharTok{@}\NormalTok{Book\{anonymous}\SpecialCharTok{:}\DecValTok{1995}\NormalTok{,}
\NormalTok{  title }\OtherTok{=}\NormalTok{ \{Handbook of Logic }\ControlFlowTok{in}\NormalTok{ Computer Science}\SpecialCharTok{:}\NormalTok{ Semantic Modelling\},}
\NormalTok{  year }\OtherTok{=}\NormalTok{ \{}\DecValTok{1995}\NormalTok{\},}
\NormalTok{  publisher }\OtherTok{=}\NormalTok{ \{Oxford University Press, Inc.\},}
\NormalTok{  address }\OtherTok{=}\NormalTok{ \{USA\},}
\NormalTok{  editor }\OtherTok{=}\NormalTok{ \{S. Abramsky and Dov M. Gabbay and T. S. E. Maibaum\},}
\NormalTok{  volume }\OtherTok{=}\NormalTok{ \{}\DecValTok{4}\NormalTok{\},}
\NormalTok{  isbn }\OtherTok{=}\NormalTok{ \{}\DecValTok{0198537808}\NormalTok{\},}
\NormalTok{\}}
\end{Highlighting}
\end{Shaded}

\textbf{Examples: InBook}

\underline{\emph{BibTeX entry}}

\begin{Shaded}
\begin{Highlighting}[]
\VariableTok{@inbook}\NormalTok{\{}\OtherTok{10}\NormalTok{.}\OtherTok{5555}\NormalTok{/}\OtherTok{218623}\NormalTok{{-}}\OtherTok{1}\NormalTok{,}
    \DataTypeTok{author}\NormalTok{       = \{Winskel, Glynn and Nielsen, Mogens\},}
    \DataTypeTok{chapter}\NormalTok{      = \{Models for Concurrency\},}
    \DataTypeTok{editor}\NormalTok{       = \{Abramsky, S. and Gabbay, Dov M. and Maibaum, T. S. E.\},}
    \DataTypeTok{title}\NormalTok{        = \{Handbook of Logic in Computer Science: Semantic Modelling\},}
    \DataTypeTok{year}\NormalTok{         = 1995,}
    \DataTypeTok{volume}\NormalTok{       = 4,}
    \DataTypeTok{isbn}\NormalTok{         = \{0198537808\},}
    \DataTypeTok{publisher}\NormalTok{    = \{Oxford University Press, Inc.\},}
    \DataTypeTok{address}\NormalTok{      = \{USA\}}
\NormalTok{\}}
\end{Highlighting}
\end{Shaded}

\underline{\emph{CFF entry}}

\begin{verbatim}
type: book
title: 'Handbook of Logic in Computer Science: Semantic Modelling'
authors:
- family-names: Winskel
  given-names: Glynn
- family-names: Nielsen
  given-names: Mogens
section: Models for Concurrency
editors:
- family-names: Abramsky
  given-names: S.
- family-names: Gabbay
  given-names: Dov M.
- family-names: Maibaum
  given-names: T. S. E.
year: '1995'
volume: '4'
isbn: 01985378082
publisher:
  name: Oxford University Press, Inc.
  address: USA
\end{verbatim}

\underline{\emph{From CFF to BibTeX}}

\begin{verbatim}
@InBook{winskelnielsen:1995,
  title = {Handbook of Logic in Computer Science: Semantic Modelling},
  author = {Glynn Winskel and Mogens Nielsen},
  year = {1995},
  publisher = {Oxford University Press, Inc.},
  address = {USA},
  editor = {S. Abramsky and Dov M. Gabbay and T. S. E. Maibaum},
  volume = {4},
  isbn = {01985378082},
  chapter = {Models for Concurrency},
}
\end{verbatim}

\hypertarget{booklet}{%
\subsubsection{Booklet}\label{booklet}}

In @Booklet \textbf{address} is mapped to location.

\begin{longtable}[]{@{}
  >{\raggedright\arraybackslash}p{(\columnwidth - 4\tabcolsep) * \real{0.33}}
  >{\raggedright\arraybackslash}p{(\columnwidth - 4\tabcolsep) * \real{0.24}}
  >{\raggedright\arraybackslash}p{(\columnwidth - 4\tabcolsep) * \real{0.42}}@{}}
\caption{Booklet model}\tabularnewline
\toprule
\begin{minipage}[b]{\linewidth}\raggedright
BibTeX
\end{minipage} & \begin{minipage}[b]{\linewidth}\raggedright
CFF
\end{minipage} & \begin{minipage}[b]{\linewidth}\raggedright
Note
\end{minipage} \\
\midrule
\endfirsthead
\toprule
\begin{minipage}[b]{\linewidth}\raggedright
BibTeX
\end{minipage} & \begin{minipage}[b]{\linewidth}\raggedright
CFF
\end{minipage} & \begin{minipage}[b]{\linewidth}\raggedright
Note
\end{minipage} \\
\midrule
\endhead
\textbf{@Booklet} & pamphlet & \\
\textbf{title*} & title & \\
\textbf{author} & authors & \\
\textbf{howpublished} & medium & \\
\textbf{address} & location & \\
\textbf{month} & month & As a fallback, \textbf{month} could be
extracted also from \textbf{date} (BibLaTeX field)/ date-published
(CFF) \\
\textbf{year} & year & As a fallback, \textbf{year} could be extracted
also from \textbf{date} (BibLaTeX field)/ date-published (CFF) \\
\textbf{note} & notes & \\
\bottomrule
\end{longtable}

\textbf{Examples}

\underline{\emph{BibTeX entry}}

\begin{Shaded}
\begin{Highlighting}[]
\VariableTok{@booklet}\NormalTok{\{}\OtherTok{booklet}\NormalTok{{-}}\OtherTok{full}\NormalTok{,}
    \DataTypeTok{title}\NormalTok{        = \{The Programming of Computer Art\},}
    \DataTypeTok{author}\NormalTok{       = \{Jill C. Knvth\},}
    \DataTypeTok{date}\NormalTok{         = \{1988{-}03{-}14\},}
    \DataTypeTok{month}\NormalTok{        = }\StringTok{feb}\NormalTok{,}
    \DataTypeTok{address}\NormalTok{      = \{Stanford, California\},}
    \DataTypeTok{howpublished}\NormalTok{ = \{Vernier Art Center\}}
\NormalTok{\}}
\end{Highlighting}
\end{Shaded}

\underline{\emph{CFF entry}}

\begin{verbatim}
type: pamphlet
title: The Programming of Computer Art
authors:
- family-names: Knvth
  given-names: Jill C.
date-published: '1988-03-14'
month: '2'
location:
  name: Stanford, California
medium: Vernier Art Center
year: '1988'
\end{verbatim}

\underline{\emph{From CFF to BibTeX}}

\begin{verbatim}
@Booklet{knvth:1988,
  title = {The Programming of Computer Art},
  author = {Jill C. Knvth},
  year = {1988},
  month = {feb},
  address = {Stanford, California},
  howpublished = {Vernier Art Center},
  date = {1988-03-14},
}
\end{verbatim}

\hypertarget{conferenceinproceedings}{%
\subsubsection{Conference/InProceedings}\label{conferenceinproceedings}}

\textbf{Examples}

\underline{\emph{BibTeX entry}}

\begin{Shaded}
\begin{Highlighting}[]
\VariableTok{@inproceedings}\NormalTok{\{}\OtherTok{inproceedings}\NormalTok{{-}}\OtherTok{full}\NormalTok{,}
    \DataTypeTok{title}\NormalTok{        = \{On Notions of Information Transfer in \{VLSI\} Circuits\},}
    \DataTypeTok{author}\NormalTok{       = \{Alfred V. Oaho and Jeffrey D. Ullman and Mihalis Yannakakis\},}
    \DataTypeTok{year}\NormalTok{         = 1983,}
    \DataTypeTok{month}\NormalTok{        = }\StringTok{mar}\NormalTok{,}
    \DataTypeTok{booktitle}\NormalTok{    = \{Proc. Fifteenth Annual ACM Symposium on the Theory of Computing\},}
    \DataTypeTok{publisher}\NormalTok{    = \{Academic Press\},}
    \DataTypeTok{address}\NormalTok{      = \{Boston\},}
    \DataTypeTok{series}\NormalTok{       = \{All ACM Conferences\},}
    \DataTypeTok{number}\NormalTok{       = 17,}
    \DataTypeTok{pages}\NormalTok{        = \{133{-}{-}139\},}
    \DataTypeTok{editor}\NormalTok{       = \{Wizard V. Oz and Mihalis Yannakakis\},}
    \DataTypeTok{organization}\NormalTok{ = \{The OX Association for Computing Machinery\}}
\NormalTok{\}}
\end{Highlighting}
\end{Shaded}

\underline{\emph{CFF entry}}

\begin{verbatim}
type: conference-paper
title: On Notions of Information Transfer in VLSI Circuits
authors:
- family-names: Oaho
  given-names: Alfred V.
- family-names: Ullman
  given-names: Jeffrey D.
- family-names: Yannakakis
  given-names: Mihalis
year: '1983'
month: '3'
collection-title: Proc. Fifteenth Annual ACM Symposium on the Theory of Computing
publisher:
  name: Academic Press
location:
  name: Boston
issue: '17'
editors:
- family-names: Oz
  given-names: Wizard V.
- family-names: Yannakakis
  given-names: Mihalis
start: '133'
end: '139'
conference:
  name: All ACM Conferences
  address: Boston
institution:
  name: The OX Association for Computing Machinery
\end{verbatim}

\underline{\emph{From CFF to BibTeX}}

\begin{verbatim}
@InProceedings{oahoullman:1983,
  title = {On Notions of Information Transfer in VLSI Circuits},
  author = {Alfred V. Oaho and Jeffrey D. Ullman and Mihalis Yannakakis},
  year = {1983},
  month = {mar},
  booktitle = {Proc. Fifteenth Annual ACM Symposium on the Theory of Computing},
  publisher = {Academic Press},
  address = {Boston},
  editor = {Wizard V. Oz and Mihalis Yannakakis},
  series = {All ACM Conferences},
  number = {17},
  pages = {133--139},
  organization = {The OX Association for Computing Machinery},
}
\end{verbatim}

\hypertarget{references}{%
\subsection*{References}\label{references}}
\addcontentsline{toc}{subsection}{References}

\hypertarget{refs}{}
\begin{CSLReferences}{1}{0}
\leavevmode\vadjust pre{\hypertarget{ref-druskat_citation_2021}{}}%
Druskat, Stephan, Jurriaan H. Spaaks, Neil Chue Hong, Robert Haines,
James Baker, Spencer Bliven, Egon Willighagen, David Pérez-Suárez, and
Alexander Konovalov. 2021. {``Citation {File} {Format}.''}
\url{https://doi.org/10.5281/zenodo.5171937}.

\leavevmode\vadjust pre{\hypertarget{ref-Haines_Ruby_CFF_Library_2021}{}}%
Haines, Robert, and The Ruby Citation File Format Developers. 2021.
{``Ruby {CFF} Library.''} Zenodo.
\url{https://doi.org/10.5281/ZENODO.1184077}.

\leavevmode\vadjust pre{\hypertarget{ref-hernangomez2021}{}}%
Hernangómez, Diego. 2021. {``{cffr}: Generate {Citation} {File} {Format}
Metadata for {R} Packages.''} \emph{Journal of Open Source Software} 6
(67): 3900. \url{https://doi.org/10.21105/joss.03900}.

\leavevmode\vadjust pre{\hypertarget{ref-lamport86latex}{}}%
Lamport, Leslie. 1986. \emph{{LaTeX}: A Document Preparation System}.
Reading, Mass: Addison-Wesley Pub. Co.

\leavevmode\vadjust pre{\hypertarget{ref-patashnik1988}{}}%
Patashnik, Oren. 1988. {``{BIBTEXTING}.''}
\url{https://osl.ugr.es/CTAN/biblio/bibtex/base/btxdoc.pdf}.

\leavevmode\vadjust pre{\hypertarget{ref-R_2021}{}}%
R Core Team. 2021. \emph{R: A Language and Environment for Statistical
Computing}. Vienna, Austria: R Foundation for Statistical Computing; R
Foundation for Statistical Computing. \url{https://www.R-project.org/}.

\leavevmode\vadjust pre{\hypertarget{ref-biblatexcheatsheet}{}}%
Rees, Clea F. 2017. {``{BibLaTeX} {Cheat} {Sheet}.''}
\url{https://osl.ugr.es/CTAN/info/biblatex-cheatsheet/biblatex-cheatsheet.pdf}.

\leavevmode\vadjust pre{\hypertarget{ref-rmarkdowncookbook2020}{}}%
Xie, Yihui, Christophe Dervieux, and Emily Riederer. 2020.
{``Bibliographies and Citations.''} In \emph{{R} Markdown Cookbook}. The
{R} Series. Boca Raton, Florida: Chapman; Hall/CRC.
\url{https://bookdown.org/yihui/rmarkdown-cookbook}.

\end{CSLReferences}

\end{document}
